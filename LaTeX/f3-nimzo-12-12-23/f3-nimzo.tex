\documentclass{article}
\usepackage{xskak}
\usepackage{listings}
\usepackage{longtable}

\title{f3 Nimzo Indian}
\date{12 December 2023}

\begin{document}
\maketitle
\tableofcontents

\section{Introduction}
\newchessgame
\mainline{1.d4 Nf6 2.c4 e6 3.Nc3 Bb4 4.f3}
\begin{center}
    \chessboard[inverse] \\
\end{center}

\section{Annotated Line: Kateryna Lagno - Hou Yifan, Rostov-on-Don FIDE GP (Women) 2011 (excerpt)}

\newchessgame[setfen=r2qr1k1/p4ppp/1pn2n2/3p4/2pP4/P1P1PPN1/1B1Q2PP/R4RK1 b - - 0 15, moveid=15b]

\begin{center}
    \chessboard[inverse] \\
\end{center}
\mainline{15...h5} $*I$ \vspace{0.2cm}\\
Black's position appears dominant, but White can prepare a strong e4 pawn break. \medskip\\
\mainline{16.Rae1} \vspace{0.2cm} \\
\variation[invar]{16.Qf2!?} *C is also good, we will see the point in Black's response.
\begin{center}
    \chessboard[inverse] \\
\end{center}
\mainline[outvar]{16...Re6?} $*I$ \vspace{0.2cm} \\
Black aims to stack on the e-file. However, Black had to calculate \variation[invar]{16...h4!} $*C$ \variation{17.Nf5 Ne7} and the pawn cannot be taken due to g5. Instead, if White had played \variation{16.Qf2!} $*C$ she could have prevented this idea due to \variation{16...h4 17.Nf5 Ne7 18.Nxh4 g5 19.Qg3!} \vspace{0.2cm} \\
\mainline[outvar]{17.Qc2 h4 18.Nf5 g6 19.Nxh4}
\begin{center}
    \chessboard[inverse] \\
\end{center}
\mainline{19...Ng4?} $I*$ \vspace{0.2cm} \\
It seems like an intuitive tactic that either wins a pawn or weakens the e4-square. \vspace{0.2cm} \\
\mainline{20.fxg4 Qxh4 21.e4! Rae8 22.e5 Qxg4 23.Re3} \\
\begin{center}
    \chessboard[inverse] \\
\end{center}
Black is positionally lost.

\section{Critical Wall: Samuel Sevian - Magnus Carlsen, MrDodgy Online Invitationals 2022}
Evaluate the following critical moves that arised or could have arisen in a sharp blitz game between Samuel Sevian and Magnus Carlsen. Magnus played the rare \movecomment{5...Bf8} in the f3 Nimzo.

\subsection{Puzzles}
\begin{center}
\begin{longtable}{cc}
\chessboard[setfen=rnbqkb1r/pp3ppp/4pn2/2p5/2PPP3/P1N5/1P4PP/R1BQKBNR w KQkq - 0 8] &
\chessboard[setfen=r2qr1k1/pp1n1ppp/5n2/2pPbN2/2P5/P2Q4/1P2N1PP/R1B2RK1 b - - 0 16] \\
\movecomment{8.Bf4} & \movecomment{16...b5} \\

\chessboard[setfen=r2qr1k1/p4ppp/3b1n2/1P3N2/2p5/P2nBQ2/1P2N1PP/R4RK1 w - - 0 21] &
\chessboard[setfen=r3rbk1/p4ppp/5n2/1P1q1N2/2pN4/P2nBQ2/1P4PP/R4R1K w - - 0 23] \\
\movecomment{21.Nxg7} & \movecomment{23.Qg3} and \movecomment{23.Nh6} \\

\chessboard[setfen=r3rbk1/p4p1p/2N2np1/1q3N2/2p5/P2nB1Q1/1P4PP/R4R1K w - - 0 25] &
\chessboard[setfen=r3rbk1/p4p1p/2N3p1/1q6/2pNn3/P2nB1Q1/1P4PP/R4R1K w - - 0 26] \\
\movecomment{25.Nd6} & \movecomment{23.Qxg6} \\

\chessboard[setfen=2r1rbk1/p4p1p/2N3p1/3q4/2pNn2Q/P2nBR2/1P4PP/5R1K b - - 0 28] &
\chessboard[setfen=4rbk1/p1r2p2/2N3p1/3q3p/2pNn2Q/P2nB2R/1P4PP/5R1K w - - 0 30] \\
\movecomment{28...Rxc6} & \movecomment{30.Rf5} \\
\end{longtable}
\end{center}
\subsection{Analysis}

\section{Critical Wall: David Anton Guijarro - Teimour Radjabov, 13th WchT Final (2022)}
Evaluate the following critical moves.

\begin{center}
\chessboard[setfen=rnbq1rk1/pp1p1ppp/4pn2/2pP4/1bP1P3/2N2P2/PP4PP/R1BQKBNR b KQ - 0 6] \\
\movecomment{6...Nxe4} \\
\chessboard[setfen=rnbqr1k1/pp3ppp/3ppn2/2pP4/1bP1P3/2N2P2/PP2N1PP/R1BQKB1R w KQ - 2 8] \\
\movecomment{8.Qd3} \\
\chessboard[setfen=rnbqr1k1/p4ppp/3ppn2/1ppP4/1bP1P3/2N2PN1/PP4PP/R1BQKB1R w KQ - 0 9] \\
\movecomment{9.Bg5} \\
\chessboard[setfen=r1bqrnk1/p4ppp/3p1n2/2pPp3/1bB1P3/2N1BPN1/PPQ3PP/R4RK1 b - - 5 13] \\
\movecomment{13...a5} \\
\chessboard[setfen=r1bqr1k1/p4ppp/3p1nn1/b1pPp3/2B1P3/P1N1BPN1/1PQ3PP/R4RK1 w - - 1 15] \\
\movecomment{15.Qf2} \\
\chessboard[setfen=1r2r1k1/p2bq1np/3p1ppB/b1pPp3/2B1P1P1/P4PN1/1P2Q2P/1R3R1K w - - 4 24] \\
\movecomment{24.a4} \\
\chessboard[setfen=3b1rk1/p2b1rnp/3p1p1q/2pP2p1/2B1PRP1/P5N1/1P1BQ1KP/5R2 w - - 0 37] \\
\movecomment{37.e5} \\
\chessboard[setfen=3b1rk1/p2b1rn1/3p1p2/2pP2pp/2B1P1Pq/P5NP/1P1BQRK1/5R2 w - - 0 39] \\
\movecomment{39.Rh1} \\
\chessboard[setfen=8/2bb1k1r/3p1p2/3P1Pp1/1R4P1/5BB1/6K1/8 b - - 0 54] \\
\movecomment{54.Kg8}
\end{center}

\section{Annotated Line: Shakhriyar Mamedyarov - Fabiano Caruana, 8th Sinquefield Cup 2021 (excerpt)}

\newchessgame[setfen=rnbq1rk1/p4ppp/3p4/2pP3n/1pP5/5P2/PP1BNKPP/R2Q1B1R b - - 2 12, moveid=12b]

\begin{center}
    \chessboard[inverse] \\
\end{center}

\mainline{12...f5!} preventing g4. \\
\mainline{13. Nf4 Nxf4 14. Bxf4 g5 15. Bc1 Nd7} \\
\variation[invar]{15...f4} was likely stronger, smothering the bishops
\begin{center}
    \chessboard[inverse, setfen=rnbq1rk1/p6p/3p4/2pP2p1/1pP2p2/5P2/PP3KPP/R1BQ1B1R w - - 0 16] \\
\end{center}
\mainline[outvar]{16. f4 Nf6 17. Kg1} \\
\begin{center}
    \chessboard[inverse] \\
\end{center}
\mainline{17...Ne4} \\
\variation[invar]{17...g4!} *C was much better. This is a very unintuitive move, and goes against the maxim to have an open position with a lead in development. However, White's position is extremely difficult to improve, while Black can dominate the e-file. \\
\mainline[outvar]{18. fxg5 f4 19. Bd3 Qe7 20. Qe2 Bf5 21. Bxf4 Qg7 $6} *C
\begin{center}
    \chessboard[inverse] \\
\end{center}

The point is after \variation[outvar]{22.Bxe4 Bxe4 23.Qxe4 Rae8}, with the help of \movecomment{Qd4} White must return the bishop. However, White can be very careful with \variation[outvar]{22.Bxe4 Bxe4 23.Bg3! $14} and the defence is adequate.

\end{document}