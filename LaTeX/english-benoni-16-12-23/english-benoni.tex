\documentclass{article}
\usepackage{xskak}
\usepackage{epigraph}

\title{Critical Chess: The English Benoni}
\date{17 December 2023}

\newcommand{\fen}[2]{\newchessgame[setfen=#1, moveid=#2]}

\newcommand{\move}[1]{\movecomment{#1}}
\newcommand{\cb}{\begin{center}\chessboard\end{center}}

% Start: 12:06am 17/12/23

\begin{document}
\tableofcontents
\maketitle
\epigraph{"I know I know nothing"}{Socrates}

\section{Introduction}
This article aims to make the reader a 2600+ player in the English Benoni. This arises after: \\

\newchessgame
\mainline{1. c4 e6 2. g3 d5 3. Bg2 Nf6 4. Nf3 Be7 5. O-O O-O 6. b3 c5 7. Bb2 Nc6 8. e3 d4 9. exd4 cxd4 10. Re1} \\
\cb

An interesting fact: Stockfish 16 at depth 30 evaluates the position as -0.1. This is irrelevant in real games compared to superior knowledge of critical ideas.

We will delve into many games and highlight ONE or TWO moment per game. We will abstract these ideas and discuss factors of the position that affect its suitability.

The issue with mainstream methods of training is they teach a few ideas deeply rather than many ideas more lightly. This ``deep'' notion is unfortuantely not a exploration of the nuances in the position that can affect the evaluation of the idea, but rather a long line of isolated moves that are only suited for the exact position. To address this, the section ``Ideas in Games'' is devoted solely to the identification of ideas - without a focus on evaluation or calculation.

\section{Ideas in Games}


\subsection{Cordova, Emilio - Banos, Oscar Francisco, Montcada op-A 24th (2016)}

\subsubsection{Sacrifice for weaknesses}

This position arises after the \move{10...Ne8} line.
\newchessgame[setfen=r1bqnrk1/pp2b1pp/5p2/8/2Pp1p2/1P4P1/PB1P2BP/RN1QR1K1 w - - 0 15, moveid=15w]
\cb

\mainline{15.Qf3 $5} *C \\
Equally as popular as \move{15.gxf4}, this move will eventually allow White to win the d4-pawn. A queen trade will likely occur, which favours White's domination over the h1-a8 diagonal.

\subsection{Adly, Ahmed - Omar, Noaman, Olympiad-40 (2012)}
\subsubsection{}

\fen{r1bqr1k1/pp3ppp/2n1pn2/2b5/2Pp4/1P3NP1/PBNP1PBP/R2QR1K1 b - - 7 12}{12b}

White has finished a \move{Nb1-a3-c2} maneuver to control b4 and d4. 
\cb
Though rare, the move \mainline{12...d3} *C is good.

\subsubsection{}

\fen{r1bqr1k1/1p1n1ppp/4p3/p1b1R3/2Pp4/1P4P1/PBNP1PBP/R2Q2K1 w - - 1 15}{15w}
\cb
\mainline{15.Rxc5} *C \\
must be considered. It is theoretically the best move here, however simplifications make it easier for Black. \move{15.Re1} was played in the game is more interesting.

\subsection{Pantsulaia, Levan - Cuenca Jimenez, Jose Fernando, Chess24 Banter Blitz Cup (2019)}

\subsubsection{}
\fen{r1bq1rk1/1p2b1pp/2n2p2/p3p3/2Pp2n1/PP1P1NP1/1B1N1PBP/R2QR1K1 w - - 0 14}{14w}
\cb
\mainline{14.b4} *C \\
The point is after \\
\mainline{14...cxb4 15.Qb3} \\
White will have the usual Benko Gambit play.

\subsubsection{}
\fen{2r1r1k1/1p1qbnpp/2n1bp2/p3p3/2PpNP1P/PP1P2P1/2Q1R1BN/2B1R1K1 b - - 0 23}{23b}
\cb
Here, \mainline{23...b5} *C should be considered.

\subsection{Wen, Yang - Li, Yankai, CHN-chT (2019)}
\subsubsection{}
\fen{1rbq1rk1/ppn1b1pp/5p2/8/2Pp1P2/1P1P4/PB1N2BP/R2QR1K1 b - - 0 17}{17b}
\cb
Black should not play \mainline{17...Ne6} due to \mainline{18.Bd5}

\subsubsection{}
\fen{1r3rk1/ppn3pp/3q1p2/2b5/2PpPP2/1P6/PB4BP/R2Q1R1K w - - 3 22}{22w}
\cb
\mainline{22.Qh5} is strong

\subsection{Moussard, Jules - Unuk, Laura, PRO League Stage (2019)}
\subsubsection{}
\fen{r1bqrbk1/pp1n1ppp/4p3/4R3/2Pp4/1P1P2P1/PB3PBP/RN1Q2K1 w - - 1 14}{14w}
\cb
\mainline{14.Re2} *C

\subsubsection{}
\fen{r1bqrbk1/pp1n1ppp/8/4p3/2Pp4/1P1P2P1/PB3PBP/RN1QR1K1 w - - 0 15}{15w}
\cb
\mainline{15.Na3} *I \mainline{15...Nc5 16.b4 Na6}

\subsection{Gashimov, Vugar - Zinchenko, Yaroslav, Wch U20 (2005)}
\subsubsection{}
\fen{3rr1k1/1p1b1ppp/1q2pn2/p1b1R3/2Pp4/PP1P1NP1/1B3PBP/R3Q1K1 w - - 7 18}{18w}
\cb
\mainline{18.Rxc5} *I \\
\variation{18.b4} *C \variation{18...axb4 19.axb4 Bxb4 20.Bxd4} is strong, though a draw is likely. \\
\variation{18.Rb1} is the best fighting move.





\end{document}