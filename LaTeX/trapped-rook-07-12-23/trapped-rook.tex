\documentclass{article}
\usepackage{xskak}
\usepackage{multicol}
\title {Trapped Rook}
\date{06 December 2023}

\begin{document}
\maketitle

\setchessboard{boardfontsize=16pt}


\section*{Introduction}
Trapping the opponent's rook can greatly increase our winning chances. However, noticing:
\begin{itemize}
  \item When we may trap a rook
  \item When should we trap a rook
  \item How to take advantage of a trapped rook
\end{itemize}
may be difficult for players. This article aims to explore these points in both practical and aesthetic dimensions. \\

\newpage

\begin{multicols*}{2}
[
\section*{The concept in isolation}
We begin with a few simple examples where the trapped rook is the focus of the position. \\
]
\newchessgame[setfen=8/8/2p1k3/p1r1p3/P7/1P2P3/1B5P/3K4 w - - 0 1]
\begin{center}
\textbf{Example 1}
\chessboard \\
\end{center}
\mainline{ 1.e4! } traps the rook. \wmove{Kd2} and \wmove{Ba3} will follow to win the rook and the game.

\newchessgame[setfen=r7/P7/8/4k3/8/7K/5BP1/8 w - - 5 3]
\begin{center}
\textbf{Example 2}
\chessboard \\
\end{center}
Black is threatening to go into a drawn endgame by sacrificing the rook for the g-pawn. However, White has the resource \mainline{1.Bg3! Kf5 2. Bb8}, and the g-pawn will advance safely. \\

The next example requires calculation to determine the evaulation of the position.
\newchessgame[setfen=5k2/r1p5/p7/P6P/B7/8/5K2/8 w - - 0 1]
\begin{center}
\textbf{Example 3}
\chessboard \\
\end{center}
\mainline {1.Bc6!} traps the rook. Black can win the pawn after \mainline{1...Ke7 2.h6 Kf6}, but White can go after Black's rook: \\
 \mainline{3. Kg3 Kg6 4. Kg4 Kxh6 5. Kf5 Kg7 6. Ke6 Kf8 7. Kd7 Kf7 8. Kc8 Ke7 9. Kb8 Kd6 10. Kxa7 Kxc6 11. Kxa6} \\
\chessboard \\
and Black is lost due to his c7-pawn.
\end{multicols*}

\newpage

\begin{multicols*}{2}
[
\section*{Sacrificing in order to trap a rook}
The evaluation of a sacrifice to trap the opponent's rook requires careful analysis of the resulting position. \\
]
Consider the following positions
\newchessgame[setfen=8/4p3/1p4k1/rb1Rp3/4P2P/P6K/1P6/8 w - - 0 1]
\begin{center}
\textbf{Example 4}
\chessboard \\
\end{center}
White must play \mainline {1.Rxb5!} to save their position. After \mainline{1...Rxb5 2. b4}, Black's rook is trapped and the game is drawn.

\newchessgame[setfen=6k1/4p3/1p4K1/rb2p3/1R2P3/P6P/1P6/8 w - - 0 1]
\begin{center}
\textbf{Example 5}
\chessboard \\
\end{center}
White is lost in position, as after \mainline{1.Rxb5 Rxb5 2.b4}, Black can afford to sacrifice his rook back: \mainline {2...Rd5!}. The rook is cannot be taken due to the e-pawn. \\
\end{multicols*}

\newpage

\begin{multicols*}{2}
[
\section*{Taking advantage of a trapped rook}
When a rook is trapped, it can be taken advantaged of by:
\begin{itemize}
  \item Playing in the opposite wing (Example 6)
  \item Winning the rook (Example 7)
\end{itemize}
]

\newchessgame[setfen=7k/8/1p6/1r1p2K1/1P1P4/2P5/8/2B5 w - - 0 1]
\begin{center}
\textbf{Example 6}
\chessboard \\
\end{center}
Black's rook is trapped, however White cannot directly win it. Instead White should play in the opposite wing until the rook is forced to move: \mainline{1.Kg6 Kg8 2.Bh6 Kh8 3.Kf7 Kh7 4. Bg7}. The rook is lost.

\newchessgame[setfen=7k/8/1p3K2/1r1p4/1P1P4/2P5/1P6/8 w - - 0 1]
\begin{center}
\textbf{Example 7}
\chessboard \\
\end{center}
White can simply head towards Black's rook in this position. \\
\end{multicols*}

\newpage
\section*{Puzzles}

\begin{center}
\begin{tabular}{cc}
\chessboard[setfen=r1k5/2p4p/p1P5/Pp2B3/1P6/8/7P/7K w - - 0 1] &
\chessboard[setfen=8/1rB5/1p2k3/1Pp5/6p1/1NP1K1Pp/7P/8 w - - 0 1] \\
1 & 2 \\

\end{tabular}
\end{center}

\end{document}